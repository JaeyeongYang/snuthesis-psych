\makeatletter
\def\input@path{{../}}
\makeatother
\documentclass[../document.tex]{subfiles}

\begin{document}

\section{서론}

% Generated from 한글 Lorem Ipsum (http://guny.kr/stuff/klorem/#/p-view)

모든 국민은 근로의 의무를 진다. 국가는 근로의 의무의 내용과 조건을 민주주의원칙에 따라 법률로 정한다. 국회나 그 위원회의 요구가 있을 때에는 국무총리·국무위원 또는 정부위원은 출석·답변하여야 하며, 국무총리 또는 국무위원이 출석요구를 받은 때에는 국무위원 또는 정부위원으로 하여금 출석·답변하게 할 수 있다. 국무총리는 대통령을 보좌하며, 행정에 관하여 대통령의 명을 받아 행정각부를 통할한다. 국정감사 및 조사에 관한 절차 기타 필요한 사항은 법률로 정한다. 대통령이 임시회의 집회를 요구할 때에는 기간과 집회요구의 이유를 명시하여야 한다. 대한민국은 통일을 지향하며, 자유민주적 기본질서에 입각한 평화적 통일 정책을 수립하고 이를 추진한다.

국회가 재적의원 과반수의 찬성으로 계엄의 해제를 요구한 때에는 대통령은 이를 해제하여야 한다. 대통령의 국법상 행위는 문서로써 하며, 이 문서에는 국무총리와 관계 국무위원이 부서한다. 군사에 관한 것도 또한 같다. 대통령은 제4항과 제5항의 규정에 의하여 확정된 법률을 지체없이 공포하여야 한다. 제5항에 의하여 법률이 확정된 후 또는 제4항에 의한 확정법률이 정부에 이송된 후 5일 이내에 대통령이 공포하지 아니할 때에는 국회의장이 이를 공포한다. 대한민국의 주권은 국민에게 있고, 모든 권력은 국민으로부터 나온다. 대통령은 조국의 평화적 통일을 위한 성실한 의무를 진다. 대통령은 국가의 독립·영토의 보전·국가의 계속성과 헌법을 수호할 책무를 진다.

한 회계연도를 넘어 계속하여 지출할 필요가 있을 때에는 정부는 연한을 정하여 계속비로서 국회의 의결을 얻어야 한다. 모든 국민은 보건에 관하여 국가의 보호를 받는다. 평화통일정책의 수립에 관한 대통령의 자문에 응하기 위하여 민주평화통일자문회의를 둘 수 있다. 국가는 과학기술의 혁신과 정보 및 인력의 개발을 통하여 국민경제의 발전에 노력하여야 한다. 모든 국민의 재산권은 보장된다. 그 내용과 한계는 법률로 정한다. 위원은 정당에 가입하거나 정치에 관여할 수 없다. 국가는 평생교육을 진흥하여야 한다. 정부는 예산에 변경을 가할 필요가 있을 때에는 추가경정예산안을 편성하여 국회에 제출할 수 있다.

\parencite{platt1964strong}

\biblio

\end{document}
